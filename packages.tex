\usepackage[
  left=2.5cm, 
  right=2.5cm, 
  top=2cm, 
  bottom=2cm, 
]{geometry}


% math mode additions
\usepackage{amsmath}
\usepackage{amsfonts}
\usepackage{amssymb}
\usepackage{mathtools} 

% nice derivatives
\usepackage[ISO]{diffcoeff}

% in-document references
\usepackage{hyperref}

% Colors, f.ex. in mdframed boxes
\usepackage[]{xcolor}

% make todo-notes
\usepackage[color=blue!40]{todonotes}

% Text boxes
\usepackage{mdframed}

% Nice eq and subsection reference
\usepackage[capitalize]{cleveref}

% Braket notation
\usepackage{braket}

% Strike through (sout)
% without normalem, it overwrites \emph :(
\usepackage[normalem]{ulem}

% to write 'goes to zero/infinity' in equatoations
\usepackage{cancel}

% for the floatbarrier macro
\usepackage{placeins}

% Drawings/feynman diagrams
\usepackage{tikz}
\usepackage{tikz-feynman}

% Tabel of contents
\usepackage{tocloft}


\usepackage[%
    style=numeric-comp, % Combine consecutive citations, i.e., [15]-[19]
    sorting=none,       % Sorts citations after their appearance in the document
    sortcites=true,     % Sorts within one "autocite", i.e., [15][17][44]
    doi=true,
    giveninits=true,    % use initials
    hyperref
    ]{biblatex}




% only subsection in toc
\setcounter{tocdepth}{2}


%%%
% New to notes

\usepackage{amsthm}
\newtheorem*{definition}{Definition}

\usepackage{slashed}
\usepackage{tikz-cd}

% adjust figures
\usepackage[export]{adjustbox}
